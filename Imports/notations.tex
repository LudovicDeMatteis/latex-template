% Basic abbreviations
%\newcommand{\st}{:}
\newcommand{\est}[1]{\widehat{#1}}
\newcommand{\err}[1]{\widetilde{#1}}

% Differential operators
\newcommand{\parDeriv}[2]{\frac{\partial #1}{\partial #2}}
\newcommand{\gradient}[2]{\nabla_{#2}{#1}}
\newcommand{\Jacobian}[2]{\gradient{#1}{#2}^T}

% Other operators
\newcommand{\kron}{\otimes}
\newcommand{\dsum}{\displaystyle\sum}

% Generic sets
\newcommand{\SE}[1]{\mathrm{SE}(#1)}
% SO group
\newcommand{\SO}[1]{\mathrm{SO}(#1)}
% so algebra
\newcommand{\soAlgebra}[1]{\mathfrak{so}(#1)}
% se algebra
\newcommand{\seAlgebra}[1]{\mathfrak{se}(#1)}
\newcommand{\gAlgebra}{\mathfrak{g}}
\newcommand{\setS}[1]{\mathcal{S}^{#1}}
\newcommand{\nullSpace}{\mathcal{N}}

% Geometric quantities

% \renewcommand{\vect}[1]{\boldsymbol{#1}}
\newcommand{\matr}[1]{\boldsymbol{#1}}
\newcommand{\refFrame}[1]{\mathcal{#1}}
\newcommand{\rotMat}[2]{{\prescript{#1}{}{\matr{R}}}_{#2}}
\newcommand{\tranVec}[3]{{\prescript{#3}{#1}{\vect{t}}}_{#2}}
\newcommand{\roll}{\phi}
\newcommand{\pitch}{\theta}
\newcommand{\yaw}{\psi}
\newcommand{\z}{\vect{z}}
\newcommand{\ez}{\vect{e}_3}
\newcommand{\eye}[1]{\matr{I}_{#1}}
\newcommand{\zeros}[2]{
\ifthenelse{\equal{#2}{1}}{\vect{0}_{#1}}{\matr{\cancel{O}}_{#1 \times #2}}
}
\newcommand{\ones}[2]{
\ifthenelse{\equal{#2}{1}}{\vect{1}_{#1}}{\matr{1}_{#1 \kron #2}}
}
% robot configuration and state
\newcommand{\world}{\refFrame{W}}
\newcommand{\body}{\refFrame{B}}
\newcommand{\bodyYaw}{\refFrame{A}}
\newcommand{\configurationVector}{\q}
\newcommand{\brm}[2]{\bcalB^{#1}_{#2}} 

% Indexes
\newcommand{\worldIndex}{\mathcal{W}}

% Weights of the Bearing Rigidity Matrix
\newcommand{\weight}{W}

\newcommand{\posElem}{p}
\newcommand{\posElemNormalized}[1]{\bar{\p}_{#1}} 
%\newcommand{\dist}{d}
\newcommand{\linVelElem}{v}
\newcommand{\linVel}{\vect{\linVelElem}}
\newcommand{\ori}{\matr{R}}
\newcommand{\angVelElem}{v}
\newcommand{\angVel}{\vect{\angVelElem}}

\newcommand{\config}{\vect{q}}
\newcommand{\linVelCmd}{\vect{u}}
\newcommand{\yawRateCmd}{w}

\newcommand{\bearing}{\vect{\beta}}
\newcommand{\projMat}{\matr{P}}
\newcommand{\skewZ}{\matr{S}}

\newcommand{\Rij}{{^{i}\!}\boldsymbol{R}_j}

\newcommand{\Rkappaj}{{^{\kappa}\!}\boldsymbol{R}_j}
\newcommand{\Rkappam}{{^{\kappa}\!}\boldsymbol{R}_m}
\newcommand{\Rupsilonj}{{^{\upsilon}\!}\boldsymbol{R}_j}
\newcommand{\Rupsilonm}{{^{\upsilon}\!}\boldsymbol{R}_m}



% Graph quantities
\newcommand{\graph}{\mathcal{G}}
\newcommand{\graphDist}{{\graph_u}}
\newcommand{\graphCompl}{{\mathcal{K}_\nRobots}}
\newcommand{\edge}{e}
\newcommand{\edges}{\mathcal{E}}
\newcommand{\edgesDist}{{\edges_u}}
\newcommand{\edgeDist}{{\edge_u}}
\newcommand{\neigh}{\mathcal{U}}
\newcommand{\vertices}{\mathcal{V}}
\newcommand{\verticesDist}{{\vertices_u}}
\newcommand{\nRobots}{N}
\newcommand{\nEdges}{{|\edges|}}
\newcommand{\nEdgesDist}{{|\edgesDist|}}

\newcommand{\rigMat}[2]{{\matr{\mathcal{B}}^{#1}_{#2}}}
\newcommand{\worldRigMat}{\bcalB^{\calW}_{\graph}}
\newcommand{\bodyRigMat}{\bcalB_{\graph}}
\newcommand{\distMat}{\matr{D}}
\newcommand{\symRigMat}[2]{\rigMat{#1}{#2}^T\rigMat{#1}{#2}}

% errors
\newcommand{\locErr}{\vect{e}_L}
\newcommand{\bearErr}{\vect{e}_F}

% Active estimation and observability
\newcommand{\unmeasElem}{\chi}
\newcommand{\unmeas}{\vect{\unmeasElem}}

\newcommand{\ObsMat}{\matr{\Omega}}
\newcommand{\fm}{\vect{f}}
\newcommand{\fu}{\vect{g}}
\renewcommand{\H}{\matr{H}}
\newcommand{\predErr}{\vect{\xi}}
\newcommand{\estErr}{\widetilde{\vect{\chi}}}
\newcommand{\obsGain}{\alpha}
\newcommand{\eig}{\lambda}
\newcommand{\lambdaBar}{\bar{\lambda}}
\newcommand{\eigsVect}{\vect{\eig}}
\newcommand{\obsIndex}{\bar\lambda}

\newcommand{\rigEig}[1]{{\eig}_{6_{#1}}}

%% useful environements
\newtheorem{lemma}{Lemma}
\newtheorem{assumption}{Assumption}
\newtheorem{remark}{Remark}
\newtheorem{theorem}{Theorem}
\newtheorem{prop}{Proposition}


%% counters
\newcounter{simulationcase}
\newcommand{\simulationCase}[1]{\refstepcounter{simulationcase}\label{#1}}
% \renewcommand\theplanMethod{\Alph{simulationcases}} 

% cleveref
%\crefname{equation}{}{}
%\Crefname{equation}{Equation}{Equations}
%\crefformat{equation}{(#2#1#3)}
%\creflabelformat{equation}{(#2#1#3)}
%\crefmultiformat{equation}{(#2#1#3}%
%  {--#2#1#3)}{, (#2#1#3)}{ and~(#2#1#3)}
%\Crefmultiformat{equation}{Equations (#2#1#3}%
%  {--#2#1#3)}{, (#2#1#3)}{ and~(#2#1#3)}
%\crefname{figure}{Fig.}{Figs.}
%\Crefname{figure}{Figure}{Figures}
%\crefname{part}{Part}{Parts}
%\Crefname{part}{Part}{Parts}
%\crefname{chapter}{Chapt.}{Chapts.}
%\Crefname{chapter}{Chapter}{Chapters}
%\crefname{section}{Sect.}{Sects.}
%\Crefname{section}{Section}{Sections}
%\crefname{subsection}{Sect.}{Sects.}
%\Crefname{subsection}{Section}{Sections}
%\crefname{subsubsection}{Sect.}{Sects.}
%\Crefname{subsubsection}{Section}{Sections}
%\crefname{appsec}{Appendix}{Appendices}
%\Crefname{appsec}{Appendix}{Appendices}
%\crefname{appendix}{Appendix}{Appendices}
%\Crefname{appendix}{Appendix}{Appendices}
%\crefname{subappendix}{Appendix}{Appendices}
%\Crefname{subappendix}{Appendix}{Appendices}
%\crefname{lemma}{Lemma}{Lemmas}
%\Crefname{lemma}{Lemma}{Lemmas}
%\crefname{remark}{Remark}{Remarks}
%\Crefname{remark}{Remark}{Remarks}
%\crefname{prob}{Problem}{Problems}
%\Crefname{prob}{Problem}{Problems}
%\crefname{constr}{Constraint}{Constraints} %{Constr.}{Constrs.}%
%\Crefname{constr}{Constraint}{Constraints}
%\crefname{algorithm}{Algorithm}{Algorithms}
%\Crefname{algorithm}{Algorithm}{Algorithms}
%crefname{prop}{Prop.}{Props.}
%\Crefname{prop}{Proposition}{Propositions}
%\crefname{ALC@unique}{step}{steps}
%\Crefname{ALC@unique}{Step}{Steps}

% Misc
\newcommand{\wrt}{w.r.t.}


%% OLD MACROS (by IROS 2016 Paper)
\newcommand{\calS}{\mathcal{S}}
\newcommand{\calW}{\mathcal{W}}
\newcommand{\calA}{\mathcal{A}}
\newcommand{\calB}{\mathcal{B}}
\newcommand{\bcalB}{\boldsymbol{\mathcal{B}}}
% \newcommand{\graph}{\mathcal{G}}
\newcommand{\calV}{\mathcal{V}}
\newcommand{\calE}{\mathcal{E}}
\newcommand{\calG}{\mathcal{G}}
\newcommand{\calK}{\mathcal{K}}
\newcommand{\calN}{\mathcal{N}}
\newcommand{\calO}{\mathcal{O}}


%\newcommand{\diag}{\mathrm{diag}}

\renewcommand{\sp}{\mathbf{span}}


\newcommand{\bbeta}{\boldsymbol{\beta}}
\newcommand{\bomega}{\boldsymbol{\omega}}
\newcommand{\bnu}{\boldsymbol{\nu}}
\newcommand{\p}{\boldsymbol{p}}
\renewcommand{\v}{\boldsymbol{v}}
\renewcommand{\o}{\boldsymbol{o}}
\renewcommand{\t}{\boldsymbol{t}}
%\newcommand{\z}{\boldsymbol{z}}
\newcommand{\e}{\boldsymbol{e}}
\newcommand{\m}{\boldsymbol{m}}
\newcommand{\n}{\boldsymbol{n}}
\newcommand{\x}{\boldsymbol{x}}
\newcommand{\y}{\boldsymbol{y}}
\newcommand{\s}{\boldsymbol{s}}
\newcommand{\g}{\boldsymbol{g}}
\newcommand{\I}{\boldsymbol{I}}
\renewcommand{\b}{\boldsymbol{b}}
\newcommand{\q}{\boldsymbol{q}}
\newcommand{\w}{\boldsymbol{w}}
\newcommand{\bpsi}{\boldsymbol{\psi}}
\renewcommand{\u}{\boldsymbol{u}}
\newcommand{\1}{\boldsymbol{1}}
\newcommand{\0}{\boldsymbol{0}}
\renewcommand{\d}{\boldsymbol{d}}
\newcommand{\f}{\boldsymbol{f}}
\newcommand{\h}{\boldsymbol{h}}
\renewcommand{\k}{\boldsymbol{k}}
\newcommand{\A}{\boldsymbol{A}}
\newcommand{\B}{\boldsymbol{B}}
\newcommand{\C}{\boldsymbol{C}}
%\renewcommand{\D}{\boldsymbol{D}}
\newcommand{\E}{\boldsymbol{E}}
\newcommand{\F}{\boldsymbol{F}}
\newcommand{\K}{\boldsymbol{K}}
\renewcommand{\L}{\boldsymbol{L}}
\newcommand{\M}{\boldsymbol{M}}
\renewcommand{\O}{\boldsymbol{O}}
\renewcommand{\P}{\boldsymbol{P}}
\newcommand{\Q}{\boldsymbol{Q}}
\newcommand{\R}{\boldsymbol{R}}
\renewcommand{\S}{\boldsymbol{S}}
\newcommand{\T}{\boldsymbol{T}}
\newcommand{\V}{\boldsymbol{V}}
\newcommand{\W}{\mathcal{W}}
\newcommand{\X}{\boldsymbol{X}}
\newcommand{\Y}{\boldsymbol{Y}}
\newcommand{\Z}{\boldsymbol{Z}}

% reference redifined
\newcommand{\refnew}[1]{~(\ref{#1})}

% Macro for the blue box in the paper.
\newcommand{\highlightBox}[2]
{
\begin{mdframed}
[backgroundcolor=green!10,rightline=true,leftline=true,linewidth=3pt]
\textbf{{#1}}

{#2}
\end{mdframed}
}

\newcommand\redBMargin[1]{
\marginpar{\begin{tikzpicture} 
\fill[red!100!white] (0,0) rectangle (0.5,0.5);
\node at (0.20,0.15) {#1};
\end{tikzpicture}}
}

\newcommand\red[1]{{\textcolor{red}{#1}}}
\newcommand\grey[1]{{\textcolor{Gray}{#1}}}
%\newcommand\green[1]{{\textcolor{green}{#1}}}
\newcommand\green[1]{{\textcolor{deepgreen}{#1}}}
\newcommand\fromPaper[1]{{\textcolor{red}{#1}}}
%\newcommand\fromPaper[1]{{\textcolor{black}{#1}}}
\newcommand\blue[1]{{\textcolor{blue}{#1}}}

% uncomment the following (and comment the following at 3 lines from here) to show the gray parts
% \definecolor{mygray}{gray}{0.75} % if you put 1 inside the previous parenthesis you will have a white, a 0 will give you a black
%\newcommand\old[1]{{\textcolor{mygray}{#1}}}
% uncomment the following (and comment the previous) to not show the gray parts
\newcommand{\old}[1]{}  %comment not showed

%% Highlight with a red box in equation
\newcommand{\highlightEqRed}[1]{%
  \colorbox{red!50}{$\displaystyle#1$}}
\newcommand{\highlightEqGreen}[1]{%
  \colorbox{green!50}{$\displaystyle#1$}}
\newcommand{\highlightEqCyan}[1]{%
  \colorbox{cyan!50}{$\displaystyle#1$}}
\newcommand{\highlightEqYellow}[1]{%
  \colorbox{yellow!50}{$\displaystyle#1$}}
%% END Highlight with a red box in equation


\newcommand\todo[1]{{\textcolor{red}{\textbf{TODO}: #1}}}
\newcommand\synonym{\textbf{{\textcolor{red}{(FIND synonim)}}}}
\newcommand\toCite{\textbf{{\textcolor{red}{(cite appropriate
        work)}}}}
%%% Local Variables:
%%% mode: latex
%%% TeX-master: "main"
%%% End:

\newcommand{\De}{\mathrm{D}}

\newcommand\gauss[2]{\mathcal{N}(#1, #2)}